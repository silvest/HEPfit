\documentclass[aps,superscriptaddress,nofootinbib,floatfix,notitlepage]{revtex4-1}
\usepackage[utf8x]{inputenc}
\pdfoutput=1
\usepackage{graphicx}
\usepackage{hyperref}
\usepackage{bm}
\usepackage{xspace}
%\usepackage[punctsep]{collref}

\newcommand{\HEPfit}{\texttt{HEPfit}\xspace}
\usepackage{lmodern}

\begin{document}

\title{\HEPfit: a Code for the Combination of Indirect and Direct
  Constraints\\ on High Energy Physics Models. \\
  v1.0: Precision Electroweak Observables, Higgs Signal Strengths,\\ and Flavour Violation in the
  Standard Model and Beyond
  \vspace*{0.5cm}
%  \begin{figure}[htb!]
%   \begin{center}
% %  \includegraphics[width=0.13\textwidth]{logo}
%   \end{center}
% \end{figure}
% \vspace*{-0.8cm}
}

\collaboration{
%\begin{figure}[h!]
  %\begin{center}
  %\includegraphics[width=0.13\textwidth]{logo}
  %\end{center}
 %\end{figure}
\HEPfit Collaboration
}
\homepage{http://www.susyfit.org} 
%\affiliation{}
\author{M.~Ciuchini}
\affiliation{INFN,  Sezione di Roma Tre, Via della Vasca Navale 84, I-00146 Roma, Italy}
\author{J.~de Blas}
\affiliation{INFN, Sezione di Roma, Piazzale A. Moro 2, I-00185 Roma, Italy}
\author{D.~Chowdhury}
\affiliation{INFN, Sezione di Roma, Piazzale A. Moro 2, I-00185 Roma, Italy}
\author{8~Eberhardt}
\affiliation{INFN, Sezione di Roma, Piazzale A. Moro 2, I-00185 Roma, Italy}
\author{M.~Fedele}
\affiliation{INFN, Sezione di Roma, Piazzale A. Moro 2, I-00185 Roma, Italy}
\author{E.~Franco}
\affiliation{INFN, Sezione di Roma, Piazzale A. Moro 2, I-00185 Roma, Italy}
\author{S.~Mishima}
\affiliation{Institute of Particle and Nuclear Studies, KEK, Tsukuba 305-0801, Japan}
\author{A.~Paul}
\affiliation{INFN, Sezione di Roma, Piazzale A. Moro 2, I-00185 Roma, Italy}
\author{L.~Reina}
\affiliation{Physics Department, Florida State University, Tallahassee, FL 32306-4350, USA}
\author{L.~Silvestrini}
\affiliation{INFN, Sezione di Roma, Piazzale A. Moro 2, I-00185 Roma, Italy}
\author{M.~Valli}
\affiliation{SISSA, via Bonomea 265, I-34136 Trieste, Italy and INFN, Sezione di Trieste, via Valerio 2, I-34127 Trieste, Italy}
\begin{abstract}
\HEPfit is a flexible tool which, given the Standard Model or any extension, allows to \textit{i)} fit the model
parameters to a given set of experimental observables;
\textit{ii)} obtain predictions for observables.
\HEPfit can be used either in Monte Carlo mode, to perform a Bayesian Markov Chain Monte Carlo
analysis of the given model, or as a library, to obtain predictions of
observables for a given point in the parameter space of the model, allowing our computational tool to be used in
any statistical framework. In the present version, Electroweak Precision Observables have been implemented
in the Standard Model and in several New Physics scenarios.


\end{abstract}
 
\maketitle

\section{Introduction}
Searching for New Physics (NP) in the LHC era requires combining
experimental and theoretical information from many sources to optimize
the NP sensitivity. NP searches, even in the absence of a positive
signal, provide useful information which puts constraints on the
viable parameter space of any NP model. Should a NP signal emerge at
future LHC runs or elsewhere, the combination of all available
information remains a crucial step to pin down the actual NP model.
NP searches at LHC require extensive detector simulations and are
usually restricted to a subset of simplified NP models. Given the high
computational demand of direct searches, it is crucial to explore only
regions of the parameter space compatible with other constraints. In
this respect, indirect searches can be helpful and make the study of
more general models viable.

\HEPfit aims at providing a tool which allows to combine all
available information to select allowed regions in the parameter space
of any NP model. To this end, it can compute many observables with
state-of-the-art theoretical expressions in a set of models which can
be extended by the user. It also offers the possibility of sampling
the parameter space using a Markov Chain Monte Carlo implemented using
the BAT library~\cite{arXiv:0808.2552}. Alternatively, \HEPfit can be
used as a library to obtain predictions of the observables in any
implemented model. This allows to use \HEPfit in any statistical
framework.

This is the first public release with a limited set of observables and
models, which we plan to enlarge. The code is released under the GNU
General Public License, so that contributions from users are possible
and welcome. In particular, the present version provides Electro Weak
Precision Observables (EWPO) in the SM and in several parametrizations
of NP contributions. In the near future, we plan to add flavour
observables and several incarnations of the Minimal Supersymmetric
Standard Model (MSSM).

The paper is organised as follows. In section~\ref{sec:Physics} we
summarize models and observables implemented in the present
version. In section~\ref{sec:Code} we describe the structure of our
code. Section~\ref{sec:Usage} contains the instructions for using our
code and some examples. In Section~\ref{sec:Comparison} we compare our
numerical results with other existing public codes.

Updated information and online documentation can be found at the
\HEPfit collaboration web site~\cite{website}.

\section{Physics Content}
\label{sec:Physics}

\subsection{Standard Model}
\label{sec:SM}
Marco Ciuchini
\subsection{Two-Higgs Doublet Models}
\label{sec:THDM}
Otto
\subsection{Minimal Supersymmetric Standard Model}
\label{sec:MSSM}
Luca
\subsection{Dimension 6 Effective Theory above the Electroweak Scale }
\label{sec:Dim6}
Jorge
\subsection{Modified Higgs Couplings}
\label{sec:Higgs}
Jorge
\subsection{Electroweak Precision Observables}
\label{sec:EWPO}
Satoshi
\subsection{Higgs Signal Strengths}
\label{sec:HSS}
Maurizio
\subsection{Flavour Observables}
\label{sec:Flavour}
Ayan, Debtosh, Luca
\begin{table}
  \centering
  \begin{tabular}{lccccc}
    Observable & SM & THDM & MSSM & Dim-6 & MHC \\
  \end{tabular}
  \caption{This table summarizes the observables and models
    implemented in the code.}
  \label{tab:summary}
\end{table}

\section{Code Description}
\label{sec:Code}
L+M+E, Ayan
\section{Usage and Examples}
\label{sec:Usage}
Ayan
\section{Comparison with Other Codes}
\label{sec:Comparison}

\subsection{Comparison with ZFITTER}
\label{sec:ZFitter}
Satoshi
\subsection{Comparison with UTfit}
\label{sec:UTfit}
L+M+E
\subsection{Comparison with JDBM}
\label{sec:JDBM}
Jorge
\subsection{Comparison with SuSEFlav}
\label{sec:Debtosh}
Debtosh
\subsection{More comparisons....}
\label{sec:more}
Otto
\cite{hep-ph/9412201,hep-ph/9908433,hep-ph/0507146,1302.1395}.


\acknowledgments
M.C. is associated to the Dipartimento di Fisica, Universit\`a di Roma
Tre. E.F. and L.S. are associated to the Dipartimento di Fisica,
Universit\`a di Roma ``La Sapienza''. We acknowledge partial support
from ERC Ideas Starting Grant n.~279972 ``NPFlavour'' and ERC Ideas
Advanced Grant n.~267985 ``DaMeSyFla''.

%\bibliography{hepbiblio}

\end{document}
